%---------Preámbulo-----------------
\documentclass[12pt]{article}

%---------Cuerpo--------------------
\begin{document}
\begin{enumerate}
\item  
%q
Dada la sucesión definida por:
\begin{equation*}
a(n) =
\begin{cases} 
  \frac{-n^2+3n-6}{4n^2-2n}  & \text{si  n es impar} \\
..............& \text{si  n es par}
\end{cases}
\end{equation*}

en caso de ser posible, completarla de modo que sea

 \begin{enumerate}
        \item convergente
        \item divergente
        \item ni convergente ni divergente
\end{enumerate}
En cualquier caso, justificar la respuesta.

\item  
%q
Para la función $f(x) = \frac{x^2-4x}{-x^2+8x-12}$
\begin{enumerate}
	\item Encontrar el conjunto de puntos para los cuales $f(x) \geq 0$ y expresarlo usando intervalos o unión de intervalos
	\item Decir cuáles son las raíces
	\item Dar el dominio de la función, expresándolo como intervalos o unión de intervalos
\end{enumerate}

\item  
%q
Considerar las siguientes  funciones:
\begin{itemize}
	\item  $f(x) = x^2-4x$
	\item  $g(x) = 2-2x$
\end{itemize}
\begin{enumerate}
	\item Representarlas en un mismo gráfico
	\item Decir en qué intervalo o intervalos $f(x) = |f(x)|$.
	\item Encontrar analíticamente los puntos del plano donde  $|f(x)| = g(x)$.
	\item Definir el conjunto de números reales para los cuales $|f(x)|< g(x)$.
\end{enumerate}

\item  
%q
Decir si las siguientes afirmaciones son Verdaderas  o Falsas. Justificar y en el caso de las afirmaciones falsas, mostrar un contraejemplo.

\begin{enumerate}
	\item Si una sucesión es oscilante no puede ser creciente.
	\item La sucesión $a(n) = 16n – n^2$ no es creciente ni decreciente.
	\item Toda sucesión creciente diverge
	\item Si una sucesión es acotada es convergente
\end{enumerate}

\item  
%q
Dada la función definida por:
\begin{equation*}
	f(x) = 
	\begin{cases} 
		\frac{1}{x^2}  & \text{si      } x < 0 \\
		x^2-3x-2& \text{si  } 0\leq x < 3\\
		(\frac{2x}{3} - 1)^2  & \text{si     } x > 3
	\end{cases}
\end{equation*}

Se pide:

\begin{enumerate}
	\item Graficar
	\item Calcular:
	\begin{itemize}
		\item  $\lim\limits_{x \to 0^+} f(x) $
		\item  $\lim\limits_{x \to 0^-} f(x) $
		\item  $\lim\limits_{x \to 3^+} f(x) $
		\item  $\lim\limits_{x \to 3^-} f(x) $
	\end{itemize}
	\item Decir si $x = 0$ y $x = 3$ son puntos de discontinuidad, y en tal caso, de qué tipo de discontinuidad de trata y por qué.
	\item Si es posible, redefinir la función (cambiarla a piaccere) en el intervalo $(-\infty, 0)$ para que la función resulte continua para todo  $k \in \mathbb{R}$ 
	\item Para la función f(x) calcular
	\begin{itemize}
		\item  $\lim\limits_{x \to +\infty} f(x) $
		\item  $\lim\limits_{x \to -\infty} f(x) $
		\end{itemize}
	\end{enumerate}
%q
\end{enumerate}

\end{document}
